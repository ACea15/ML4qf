% Created 2023-07-31 Mon 12:17
% Intended LaTeX compiler: pdflatex
\documentclass[11pt]{article}
\usepackage[utf8]{inputenc}
\usepackage[T1]{fontenc}
\usepackage{graphicx}
\usepackage{longtable}
\usepackage{wrapfig}
\usepackage{rotating}
\usepackage[normalem]{ulem}
\usepackage{amsmath}
\usepackage{amssymb}
\usepackage{capt-of}
\usepackage{hyperref}
\usepackage[margin=1in]{geometry}
\usepackage{mathtools}
\author{Alvaro Cea}
\date{\today}
\title{Portfolio Construction using Black-Litterman Model and Factors}
\hypersetup{
 pdfauthor={Alvaro Cea},
 pdftitle={Portfolio Construction using Black-Litterman Model and Factors},
 pdfkeywords={},
 pdfsubject={},
 pdfcreator={Emacs 28.1 (Org mode 9.5.2)}, 
 pdflang={English}}
\begin{document}

\maketitle
\tableofcontents


\section{Introduction}
\label{sec:orgd6e3095}
\subsection{Develop computational tools and external libraries}
\label{sec:org4e2df7b}
\begin{itemize}
\item getFamaFrenchFactors
\url{https://pypi.org/project/getFamaFrenchFactors/}
\end{itemize}
\section{Theory}
\label{sec:org2946104}
\subsection{Fama-French Model}
\label{sec:orge8cd720}

It is one of the multi-factor models which is widely used in both academia and industry to estimate the excess return of an investment asset. It is an extension to Capital Asset Pricing Model (CAPM) by adding two additional factors apart from the market risk when estimating the excess returns of an asset. The three factors considered in this model are:

Market factor (MKT) — Excess market return
Size factor (SMB) — Excess return with a small market cap over those with a large market cap
Value factor (HML) — Excess return of value stocks over growth stocks.

The Fama-French model is widely known as a stock market benchmark to evaluate investment performance.

\section{Results}
\label{sec:org419c8e0}
\subsection{Portfolio and Factor analysis}
\label{sec:org68f556d}
\subsubsection{Asset selection}
\label{sec:orgb26da2f}
\subsubsection{Factor collection}
\label{sec:org08cce20}
\begin{verbatim}
                            OLS Regression Results                            
==============================================================================
Dep. Variable:                      y   R-squared:                       0.608
Model:                            OLS   Adj. R-squared:                  0.587
Method:                 Least Squares   F-statistic:                     28.98
Date:                Mon, 31 Jul 2023   Prob (F-statistic):           1.92e-11
Time:                        12:17:38   Log-Likelihood:                 126.50
No. Observations:                  60   AIC:                            -245.0
Df Residuals:                      56   BIC:                            -236.6
Df Model:                           3                                         
Covariance Type:            nonrobust                                         
==============================================================================
                 coef    std err          t      P>|t|      [0.025      0.975]
------------------------------------------------------------------------------
const          0.0155      0.004      3.671      0.001       0.007       0.024
Mkt-RF         0.8628      0.094      9.194      0.000       0.675       1.051
SMB           -0.3160      0.152     -2.084      0.042      -0.620      -0.012
HML           -0.3282      0.109     -3.022      0.004      -0.546      -0.111
==============================================================================
Omnibus:                        2.381   Durbin-Watson:                   2.596
Prob(Omnibus):                  0.304   Jarque-Bera (JB):                1.661
Skew:                           0.160   Prob(JB):                        0.436
Kurtosis:                       3.750   Cond. No.                         39.9
==============================================================================

Notes:
[1] Standard Errors assume that the covariance matrix of the errors is correctly specified.
\end{verbatim}

\subsubsection{P\&L and backtesting}
\label{sec:orgf9ea864}
\subsection{Black-Litterman implementation}
\label{sec:org9f535a6}
\subsubsection{Prior and posterior returns construction}
\label{sec:orge69f300}
\subsubsection{Views on}
\label{sec:org55d708d}
\subsubsection{Covariance treatment}
\label{sec:org9a50ea6}
\subsubsection{Portfolio weights optimisation}
\label{sec:orgf604020}
\subsubsection{Analysis and discussion}
\label{sec:org8434619}
\subsubsection{Performance comparison}
\label{sec:orgac02026}
\end{document}